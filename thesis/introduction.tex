\chapter{Introduction}
One of the general goals of artificial intelligence is to build
computing devices that are ``context-aware'', that act as
more than just passive number-crunching machines that receive input data through very
restrictive and wholly human-operated channels such as a keyboard or mouse.
Context-aware devices are capable of using sensor data to understand the
environment that they are situated in, such as the locations of nearby objects
and how the objects are moving \cite{abowd99}. One subfield of context-aware
computing that has been receiving considerable attention in recent years is activity
detection. The goal of activity detection is to build computer plus sensor systems
that are able to determine what activity a human subject is performing at any given
moment.

Such systems have a variety of real-world applications. Research is 
exploring the feasibility of using both wearable and non-wearable sensor systems to
monitor the health of elderly patients that have or are at risk of developing
degenerative physical and mental diseases \cite{fogarty06}. The goal is to 
build sensor-based monitoring systems that can aid doctors and family
members in tracking the decline of patients over time. Also, detection of an
abnormal activity may indicate that a senior is undergoing a serious medical
event such as a heart attack or slip-and-fall \cite{wang12}.
Another application of activity detection is to track the energy expenditure
of subjects as they go through the course of their day. The
traditional method of performing such tracking is with self-reporting by the
subject of their activities. Wearable sensors offer an alternative approach that
is not susceptible to misreporting due to bias, poor memory, or other
confounding factors that a human reporter introduces into the system. One
approach is to use sensor data to estimate the vigorousness or metabolic equivalent
(MET) of an activity and calculate energy expenditure directly \cite{staudenmeyer09},
while another is to attempt to predict the type of activity performed, and calculate energy
expenditure using knowledge of how vigorous that activity is generally \cite{trost12}.

Activity detection generally assumes that sensor data will be represented as a
time series, and that at any given moment in the time series the subject is
performing one and only one type of activity. Thus the time series is thought
of as being partitioned into a number of non-overlapping intervals (\emph{windows}), which are
delimited by moments in time when the subject stopped performing one activity
and started performing another. Previous work has treated activity detection as an
offline problem, and has rarely considered performance metrics other than accuracy.
In this work we are interested in the feasibility of partitioning and
classifying a time series on free-living data in real time. In addition to
accuracy, we will also evaluate our algorithms in terms of the amount of time
required to detect that an activity change has occurred.

We used change-point detection to partition time series data into activity windows for classification.
Change-point detection is a field of statistics popular in control theory and other similar
applications. We call this our top-down approach, because this
method takes as input an initial time series and partitions it into smaller pieces using
change-point detection. As an alternate approach we
used the well-known technique of partitioning the time series into small fixed-length,
non-overlapping windows, predicting the activity type of each window
using a base classifier, treating that prediction as the observable
state of an HMM, and then finally solving the HMM for its hidden states. We call
this our bottom-up approach, because we begin with small windows of fixed-length,
and use an HMM to smooth windows together into larger activity intervals.
