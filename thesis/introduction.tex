\chapter{Introduction}
One of the general goals of the field of artificial intelligence is to build
computing devices that are ``context-aware'', that act as
more than just passive number-crunching machines that receive input data through very
restrictive and wholly human-operated channels such as a keyboard or mouse.
Context-aware devices are capable of using sensor data to understand the
environment that they are situated in, such as the locations of nearby objects
and how the objects are moving \cite{abowd99}. One subfield of context-aware
computing that has been receiving a lot of attention in recent years is activity
detection. The goal of activity detection is to build computer + sensor systems
that are able to determine what activity a human subject is performing at any given
moment.

Such systems have a variety of real-world applications. Researchers have been
exploring the feasibility of using both wearable and non-wearable sensor systems to
monitor the health of elderly patients that have or are at risk of developing
degenerative physical and mental diseases \cite{fogarty06}. The hope is that 
eventually sensor-based monitoring systems will aid doctors and family
members in tracking the decline of subjects over time. Also detection of an
abnormal activity may indicate that a senior is undergoing a serious medical
event such as a heart attack or slip-and-fall \cite{wang12}. A second
application of activity detection is the studying of energy expenditure
of subjects as they go through the course of their day. The
traditional method of performing such tracking is with self-reporting by the
subject of their activities. Wearable sensors offer an alternative approach that
is not susceptible to misreporting due to bias, poor memory, or other
confounding factors that a human reporter introduces into the system. One
approach is to estimate the vigorousness or metabolic equivalent
(MET) of an activity and calculate energy expenditure directly, while another
is to attempt to predict the type of activity performed, and calculate energy
expenditure using knowledge of how vigorous that activity is on the average
\cite{staudenmeyer09} \cite{trost12}.

Activity detection generally assumes that sensor data will be represented as a
time series, and that at any given moment in the time series the subject is
performing one and only one type of activity. Thus the time series is thought
of as being partitioned into a number of non-overlapping intervals (\emph{windows}), which are
delimited by moments in time where the subject stopped doing one activity and
started doing another. In this work we address the problem of performing the
partition on unlabeled sensor data, i.e. deciding at which point an activity ends and 
thereby the next activity begins. We use change-point detection, which is a
field of statistics popular in control theory and other similar applications,
to predict changes in activity. We call this our top-down approach because we
begin with an initial time series and split it into smaller pieces using
change-point detection.
As a baseline comparison to this method we
use the well known technique of splitting the time series into small fixed-
length non-overlapping windows, predicting the activity type of each window
using a base classifier, treating that prediction as the observable
state of an HMM, and finally solving the HMM for its hidden states. We call
this our bottom-up approach, because we begin with small windows of fixed-
length, and use an HMM to smooth and aggregate those windows together into
larger activity intervals.
