\section{Top-Down Approach}
\label{sec:topdown}

\subsection{Change Point Detection}
For this approach, the data was split into non-overlapping segments for
featurization using techniques from the statistical field of change point detection.
Change point detection has found application in many problem domains that require analysis of time series data
from dynamic systems, including failure detection \cite{bae13}, quick detection of
attacks on computer networks \cite{tartakovsky06}, and monitoring of heartbeat fluctuations during
sleep \cite{staudacher05}. Change point detection assumes that each tick of a time series is a draw from some
process, but that the process may suddenly change as time passes.
The goal is to predict when these changes have occurred.
A \emph{score} is generated for each time tick, and if the score is
above a given threshold a change is predicted to have occurred between that tick
and its immediate predecessor. To generate a score at
a time tick, a window of data that immediately preceeds it (the
\emph{reference data}) is compared to it along with a window of data that immediately follows it
(the \emph{test data}).

\begin{figure}
 \centering
 \includegraphics[scale=0.8]{cpd_ref_test_new.png}
 \caption{Change Point Detection}
 \label{fig:cpd_ref_test.png}
\end{figure}

Model-based approaches to change point detection assume that each tick in
a time series is a draw from some underlying probability distribution.
Scores are generated by estimating the distribution of the reference data
and the test data, and then by calculating the likelihood
that the two distributions are different.
Where it is reasonable to assume that the data belongs to a particular
family of distributions then parametric estimation methods have been employed
\cite{thatte11}. If no such modeling assumptions are reasonable then 
non-parametric methods have also been found to be viable \cite{matteson12}.
Distance-based approaches such as Singular Spectrum Analysis
generate scores through other metrics of 
dissimilarity or difference between the reference data and the test data
\cite{moskvina03}.
Notationally, we say that for each tick $i$ in a time series:

\[
s_i = D(x_{r,i}, x_{t,i})
\]

Where $s_i$ is the score of the $i$th tick, $x_{r,i}$ is the reference
data associated with the $i$th tick, $x_{t,i}$ is the test data
associated with the $i$th tick, and $D(A,B)$ is a function that computes the
dissimiliarity between a matrix of data $A$ and matrix of data $B$, and varies
between change point algorithms. Note that for a given
algorithm it may not be possible to generate scores right at the very beginning
of the time series (insufficient reference data) or at the very end of a time
series (insufficient test data).

\subsection{Experimental Setup}

There are many different modeling assumptions and associated algorithms
for generating change point detection scores, and one simple baseline approach that we 
wanted to test was the Shewhart Control Chart. This approach assumes that the reference data is drawn from a
multivariate normal distribution, and that scores are calculated by the Mahalanobis
distance of the target time tick from the estimated multivariate normal:

\[
s_i = \sqrt{(\bar{x}_{r,i} - x_i)^T \, S_{r,i}^{-1} \, (\bar{x}_{r,i} - x_i)}
\]

where $\bar{x}_{r,i}$ is the sample mean of the reference data, $S_{r,i}$
is the sample covariance matrix of the reference data, and $x_i=x_{t,i}$ is
the $i$th data point \cite{shewhart26}.

We were also interested in testing the performance of a newer and more
sophisticated change point detection algorithm: the
Kullback-Leibler Importance Estimation Procedure (KLIEP),
introduced by Kawahara and Sugiyama \cite{sugiyama09} \cite{sugiyama08}.
This approach generates scores using the Kullback-Leibler (KL)
divergence between the reference data and the test data. One method of doing this
is to estimate the density of the reference distribution and test distribution
separately, and then compare them using a likelihood ratio
(known in the change point detection literature as \emph{importance}). 
Instead, KLIEP estimates the importance directly using a non-parametric model.

Let the estimate of the importance $\hat{R}$ be represented by this model:

\[
\hat{R} = \frac{p_{t}}{\hat{p}_{r}} = \sum_{j=1}^{n_{t}} \alpha_j K_G(x,x_{t,j})
\]

Where $p_{r}$ and $p_{t}$ are the probability densities of the reference data and the test
data, $n_{t}$ is the number of ticks in the test window, $\alpha$ is a
vector of model parameters to solve for, $x$ is the concatenation of the reference and the
test data, $x_{t,j}$ is the $j$th element of the test data,
and $K_G(A,B)$ is the Gaussian kernel with width $\sigma$:

\[
K_G(A,B) = \exp \left(-\frac{||A-B||^2}{2\sigma^2}\right)
\]

Now solve for $\alpha$ so that the empirical KL divergence between $\hat{p}_{t}$ and
$p_{t} = p_{r}\hat{R}$ is minimized, which is equivalent to the following convex optimization
problem:

\[
\begin{dcases}
 \max_{\alpha} \quad \sum_{j=1}^{n_t} \, \log \left( \sum_{k=1}^{n_t} \alpha_k K_G(x_{t,j}, x_{t,k}) \right) \\
 \,\, \text{s.t.} \quad\, \frac{1}{n_r} \sum_{j=1}^{n_r} \sum_{k=1}^{n_t} \alpha_k K_G(x_{r,j},x_{t,k}) = 1 \\
 \qquad \quad \text{and} \; \alpha_1 \ldots \alpha_{n_t} \ge 1
\end{dcases}
\]

Finally, the scores that we wish to generate are just the estimate of the importance given by the
solution to the complex optimization problem, i.e. $s_i = \hat{R}_i$.

Since this approach uses a Gaussian kernel, it requires the selection of
a kernel width $\sigma$ for each time tick. We used an implementation of
KLIEP that is available at Sugiyama's website, which included a cross-validation
procedure for the value of $\sigma$. The CV procedure chooses a number of disjoint
splits of the test data along with a number of different candidate $\sigma$'s, and runs
KLIEP with each combination of split and candidate $\sigma$. Then it chooses the candidate $\sigma$
that, on the average across all of the splits, maximizes the KL divergence (the
$\max_{\alpha}$ equation above) the most.

For the OSU Hip dataset, we used this CV procedure io choose the the kernel width at each individual time tick. This computationally
intensive approach was impractical for the UQ dataset because it is orders of magnitude larger,
so instead of running it on every tick of that data, we ran the CV procedure on a number of
random ticks drawn from the data. From this we were able
to empirically identify 0.01 as a plausible $\sigma$, and so fixed $\sigma$
at that value for our experiments on that dataset.

Previous research \cite{zheng12} found that on the average a window size of 10
seconds contained just enough information to discriminate well between OSU Hip activities.
Since this window size worked well in previous experiments, and since the 
activities of the UQ dataset were comparable in average length, we decided to
fix our reference window size at 10 seconds for both datasets. Because we were interested in
minimizing detection time, and because 1 second was the smallest window that
we felt could provide some information about an activity, we fixed our test
window size at 1 second for both datasets.

Once scores were generated, we tested a number of threshold values that determined which
scores were high enough to be considered a predicted change-point.
Threshold values were chosen by considering the false positive rates of
change prediction for the change point detection algorithms. A smaller false positive rate
corresponded to a higher and more conservative threshold, which split the
time series into fewer segments for featurization. A larger false positive rate
corresponded to a lower threshold, which split the time series into more segments.
