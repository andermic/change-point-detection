\chapter{Related Work}
As mentioned in the previous chapter, sensor systems may consist
of environmental or wearable devices. Some examples of environmental
sensors that activity detection researchers have used to
gather data are microphones \cite{fogarty06}, weight detection
panels \cite{rowan05}, cameras \cite{duong05}, and water usage detectors
\cite{fogarty06}. Researchers will generally place environmental sensors
inside of a house, have subjects live in the house for a period of time, and
attempt to predict for activity types like cooking, watching TV, etc.

Various wearable devices have been tried as well, such as RFID gloves
\cite{gu09} \cite{rowan05}, but the most popular wearable for activity detection
purposes is the accelerometer. Besides being inexpensive, accelerometers
tend to be small and lightweight, and so are fairly unobtrusive and
user-friendly. Accelerometers also gather data at a high frequency, and as such
may be used to collect a sizeable amount of data in a relatively short amount
of time.

Whether or not an accelerometer will yield data that is discriminative for a
set of activity types depends partially on where the accelerometer is worn on
a subject's body. For example, an accelerometer worn on the ankle will be
more discriminative for the activity of cycling than it would be if it was worn
on the hip, and different types of arm movements will likely be
discriminated only by an accelerometer worn on the arm. For this reason some
researchers have opted to use multiple accelerometer systems to capture
movement information from different parts of the body \cite{bao04}
\cite{devries11}. However, this approach can be cumbersome for the wearer, so a
single accelerometer is preferred when it is reasonable to assume that it will
be discriminative for the relevant set of activities. Recent research has
noticed that real subjects are likely to carry smartphones with built-in
accelerometers, and has explored the possibility of collecting data from those
accelerometers for activity detection purposes \cite{bao04} \cite{choudhury08}
\cite{kwapitz10} \cite{rai12}.

Activity sensor data tends to be noisy and not amenable to a
deterministic or rule-based analysis, so activity types are typically modeled
probabilistically, and activity detection is usually formulated as a supervised
learning problem. The various common supervised learning algorithms are all
familiar to the activity detection literature, though neural networks are
especially popular, such as in \cite{aminian95} \cite{song07} \cite{staudenmeyer09}.
More complicated modeling approaches have also been tried, such as
plurality voting with bagged, boosted, and stacked classifiers \cite{ravi05};
conditional random fields \cite{blanke10} \cite{gu09} \cite{vankasteren08}
\cite{wu09}; and HMMs \cite{gu09} \cite{lester05} \cite{pober06} \cite{wu09}.

In the past few years researchers have begun to recognize the need to test on
realistic free-living data \cite{gu09} \cite{kwapitz10} \cite{strohrmann11}
\cite{wu09}. The time required to detect that a change in activity has occurred
was considered in \cite{grauman12} \cite{song06}, but in the context of
the very different problem of video activity recognition. Our work is one of the
first to consider the feasibility of performing accelerometer activity
recognition in real time by using both accuracy and detection time as
performance metrics.
