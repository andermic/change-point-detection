\chapter{Related Work}
As mentioned in the previous chapter sensor systems may consist
of environmental or wearable devices. Some examples of environmental
sensors that activity detection researchers have used to
gather data are microphones \cite{fogarty06}, weight detection
panels \cite{rowan05}, cameras \cite{duong05}, and water usage detectors
\cite{fogarty06}. Researchers will generally place environmental sensors
along with their subjects inside of a house, and attempt to predict
for activity types like various household chores, etc.

Various wearable devices have been tried as well, such as RFID gloves
\cite{rowan05}, but the most popular wearable for activity detection
purposes is the accelerometer. Besides being inexpensive, accelerometers
tend to be small and lightweight, and so are fairly unobtrusive and user-
friendly. Accelerometers also gather data at a high frequency, and as such make
for a good way to get ahold of a sizeable amount of data in a fairly short
amount of time.

Whether or not an accelerometer will be discriminative for a
set of activity types depends partially on where the accelerometer is worn on
the subjects' body. For example, an accelerometer worn on the ankle will be
more discriminative for the activity of cycling than it would be if it was worn
on the hip, and different types of arm movements will likely be
discriminated only by an accelerometer worn on the arm. For this reason some
researchers have opted to use multiple accelerometer systems to capture
movement information from different parts of the body \cite{bao04}
\cite{devries11}.

Sensor data from activities tends to be noisy and not very amenable to a
deterministic or rule-based analysis, so activity types are typically modeled
probabilistically, and activity detection is usually formulated as a supervised
learning problem. The various common supervised learning algorithms are all
familiar to the activity detection literature though neural networks are
especially popular, such as in \cite{aminian95} \cite{song07} \cite{staudenmeyer09}.
More complicated metamodeling schemes have also been attempted, such as plurality voting with bagged,
boosted, and stacked classifiers \cite{ravi05}; and the use of conditional
random fields \cite{vankasteren08}.
